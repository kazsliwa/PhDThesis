\thispagestyle{fancy}

\textrm{}\\\\
\noindent\textbf{\huge\textsf{Abstract}}\\\\

\noindent In this thesis, we present theoretical and statistical
techniques broadly related to systems of dynamically-interacting
particles.  We apply these techniques to observations of dense 
star clusters in order to study gravitational interactions between
stars.  These include both long- and short-range interactions, as well
as encounters leading to direct collisions and mergers.  The latter
have long been suspected to be an important formation channel for
several curious types of stars whose origins are unknown.  The former
drive the structural evolution of star clusters and, by leading to
their eventual dissolution and the subsequent dispersal of their
stars throughout the Milky Way Galaxy, have played an important role
in shaping its history.  Within the last few decades, theoretical work
has painted a comprehensive picture for the evolution of star
clusters.  And yet, we are still lacking direct observational
confirmation that many of the processes thought to be driving this
evolution are actually occuring.  The results presented in
this thesis have connected several of these processes to real
observations of star clusters, in many cases for the first time.  This
has allowed us to directly link the observed
properties of several stellar populations to the physical
processes responsible for their origins.
%  Given the importance of star
%clusters for stellar birthing, this also represents a key step toward
%re-constructing the history of our Galaxy.

We present a new method of quantifying the frequency of encounters
involving single, binary and triple stars using an adaptation of the
classical mean free path approximation.  With this technique, we have
shown that dynamical encounters involving triple stars occur commonly
in star clusters, and that they are likely to be an important
dynamical channel 
for stellar mergers to occur.  This is a new result that has important
implications for the origins of several peculiar types of stars (and
binary stars), in
particular blue stragglers.  We further present several new
statistical techniques that are broadly applicable to systems of
dynamically-interacting particles composed of several different types
of populations.  These are applied to observations of star clusters in
order to obtain quantitative constraints for the degree to 
which dynamical interactions affect the relative sizes and spatial
distributions of their different stellar populations.  To this end, 
we perform an extensive analysis of a large sample of colour-magnitude
diagrams taken from the ACS Survey for Globular Clusters.  The results
of this analysis can be summarized as follows:  (1) We have compiled a
homogeneous catalogue of stellar populations, including main-sequence,
main-sequence turn-off, red giant branch, horizontal branch and blue
straggler stars.  (2) With this catalogue, we have quantified the
effects of the cluster dynamics in determining the relative sizes
and spatial distributions of these stellar populations.  (3) These
results are particularly interesting for blue stragglers since they
provide compelling evidence that they are descended from binary stars.
(4) Our analysis of the main-sequence populations is consistent with 
a remarkably universal initial stellar mass function in old massive
star clusters in the Milky Way.  This is a new result with 
important implications for our understanding of star formation in the
early Universe and, 
more generally, the history of our Galaxy.  Finally, we describe how
the techniques presented in this thesis are ideally suited for
application to a number of other outstanding puzzles of modern
astrophysics, including chemical reactions in the interstellar medium
and mergers between galaxies in galaxy clusters and groups.  
%These include the role
%played by the coagulation of dust grains in the interstellar medium
%in the early stages of both star and planet formation, as well as
%the importance of mergers and collisions between galaxies for
%understanding their morphologies.  


\newpage
\thispagestyle{empty}
\mbox{}
%to get the Dedication of the right-hand side