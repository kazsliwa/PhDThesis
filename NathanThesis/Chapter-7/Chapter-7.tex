\pagestyle{fancy}
\headheight 20pt
\lhead{Ph.D. Thesis --- N. Leigh }
\rhead{McMaster - Physics \& Astronomy}
\chead{}
\lfoot{}
\cfoot{\thepage}
\rfoot{}
\renewcommand{\headrulewidth}{0.1pt}
\renewcommand{\footrulewidth}{0.1pt}

\chapter{Summary \& Future Work} \label{chapter7}
\thispagestyle{fancy}

%In this thesis, we have presented theoretical and statistical
%techniques broadly related to systems of dynamically-interacting
%particles.  We have applied these methods to observations of dense
%star clusters in order to study gravitational interactions between
%stars.  %These include both long- and short-range interactions, as well
%as encounters leading to direct collisions and mergers.  
%These
%processes have been described in detail in Chapter~\ref{chapter1}.
%They can be summarized as follows.  
%
%Within stellar communities, the life of a star can be dramatically
%impacted by its peers.  The interactions are typically peaceful and
%develop gradually over time.  However, violent encounters also occur
%and what can be thought of as a shoving match ensues.  The skirmish
%usually ends abruptly when one or more of the stars are forcefully
%propelled from the
%encounter.  From time to time, however, stars become entirely devoured
%by their opponents.  
%The effects of all of these interactions reverberate throughout
%clusters, stimulating their
%maturation.  Society evolves and adapts to accomodate the
%ever-changing demands its citizens impose upon it.
%
%Given the considerable evolution in our understanding of star clusters
%within the past few decades, what can be summarized as the study of
%stellar psychology has made considerable progress, particularly on the
%theoretical front.  That is, 
Within the last few decades, theoretical work has painted a
comprehensive picture for the 
various forms of gravitational interactions operating within star
clusters.  
And yet, direct observational confirmation that many of these processes
are actually occurring is still lacking.  %, and the 
%consequences of these different manifestations of stellar psychology.
The results presented in 
this thesis have connected several of these processes to real
observations of star clusters, in many cases for the first time.  This
has allowed us to directly link the observed
properties of several peculiar stellar populations to the physical
processes responsible for their origins.  Given the importance of star
clusters for star formation, this also represents a key step toward
re-constructing the history of our Galaxy.  
%Our results have advanced our understanding of stellar dynamics,
%stellar evolution and, in particular, the interplay that occurs
%between the two in dense star clusters. 

In Chapter~\ref{chapter2}, we have presented a new adaptation of the
mean free path approximation.  Our technique provides analytic
time-scales for the rate of close gravitational encounters between
single, binary and triple stars in dense star systems.  With it, we
showed that encounters involving triple stars occur more commonly than
any other type of dynamical interaction in at least some star
clusters, and that these could be the dominant dynamical production
mechanism for stellar mergers.  This is a new result with important
implications for both star cluster evolution and the formation of
several types of stellar exotica.  Our method
can be generalized for application to systems composed of several
different types of particles that evolve under the influence of any
force-mediated interactions.  For example, our technique is
well suited for application to collisions between atoms and dust
grains in the interstellar medium (ISM) \citep[e.g.][]{spitzer41a,
  spitzer41b, spitzer42}.  In this case, the relevant forces governing 
the dynamics are electromagnetic instead of gravitational, however the
general form adopted for the mean free path approximation remains the same.
Specifically, we must only replace the gravitationally-focused
collisional cross-sections with their charge-focused analogues.
Recent observations have provided 
estimates for the concentrations of the different constituents of the
ISM, in addition to their temperatures \citep[e.g.][]{delburgo03,
  kiss06}.  These data provide the required ingredients %for the
%application of our technique to the dynamics of the ISM.  This would yield
to obtain analytic estimates for the rates of collisions between
atoms, molecules, as well as both small and large dust grains in the
ISM, and would allow us to make
predictions for the evolution of their relative concentrations.  This
is an important step in the development of our understanding of dust
grain coagulation and, by extension, the initial phases of star
formation in dense molecular clouds \citep[e.g.][]{mckee07}, the
interaction of winds from
evolved stars with the surrounding ISM \citep[e.g.][]{glassgold96},
and planet formation in protoplanetary disks \citep[e.g.][]{absil10}.

In Chapter~\ref{chapter3}, we introduced a statistical technique to
compare the relative sizes of different populations of stars in a
large sample of star clusters.  %Our method was used to quantify
%how these distributions change with increasing radial distance from
%the cluster centre.  
We refined this technique in 
Chapter~\ref{chapter4}, and applied it to a large sample of clusters
collected using state-of-the-art observations.  Our results suggest
that dynamical effects do not significantly affect the relative sizes
of the different stellar populations.  This is the case for all
stellar populations above the main-sequence turn-off in the cluster
CMD.  Blue stragglers alone present a possible exception to this,
since we have identified a general trend in which more massive
clusters are home to proportionately smaller blue straggler
populations.  This provides compelling evidence in favour of a binary origin
for blue stragglers in even the dense 
cores of massive star clusters where direct collisions between stars are
expected to occur frequently \citep{knigge09}.  Although we have applied these
techniques to observations of dense star clusters throughout 
this thesis, they can be generalized for application to a number of
other studies related to population statistics.  For example, massive
elliptical galaxies have been identified in Galaxy Clusters and
Groups, and these could have formed from the mergers of smaller
galaxies.  Observational studies have now been 
performed that are allowing for the compilation of catalogues
providing population statistics for several different galaxy types in
a large number of Galaxy Clusters and Groups \citep[e.g.][]{abell58,
  abell89}.  Our technique is
well suited for application to these data, and would allow for a
systematic comparison between the relative population sizes of the
different galaxy types.  Among other things, this
would help to constrain the origins of massive elliptical galaxies in
Galaxy Clusters in
much the same way we have constrained the origins of BSs in globular
clusters.
% in a large sample of Galaxy Clusters and
%Groups.

%In Chapter~\ref{chapter3}, we have presented a new adaptation of the
%mean free path approximation.  Our technique provides analytic
%time-scales for the rate of close gravitational encounters between
%single, binary and triple stars in dense star systems.  With it, we
%showed that encounters involving triple stars occur more commonly than
%any other type of dynamical interaction in at least some star
%clusters, and that these could be the dominant dynamical production
%mechanism for stellar mergers.  This is a new result with important
%implications for both star cluster evolution and the formation of
%several types of stellar exotica.  Our method
%can be generalized for application to systems composed of several
%different types of particles that evolve under the influence of any
%force-mediated interactions.  For example, our technique is
%ideally suited for application to collisions between atoms and dust
%grains in the interstellar medium (ISM) \citep[e.g.][]{spitzer41a,
%  spitzer41b, spitzer42}.  Recent observations have provided
%estimates for the concentrations of the different consitituents of the
%ISM, in addition to their temperatures \citep[e.g.][]{delburgo03,
%  kiss06}.  These data provide the required ingredients for the
%application of our technique to the dynamics of the ISM.  This
%would yield analytic estimates for the rates of collisions between 
%atoms, molecules, as well as both small and large dust grains in the
%ISM, and would allow us to make
%predictions for the evolution of their relative concentrations.  This
%is an important step in the development of our understanding of dust
%grain coagulation and, by extension, the initial phases of star
%formation in dense molecular clouds \citep[e.g.][]{mckee07}, the
%interaction of winds from
%evolved stars with the surrounding ISM \citep[e.g.][]{glassgold96},
%and planet formation in protoplanetary disks \citep[e.g.][]{absil10}.
%
In Chapter~\ref{chapter5}, we present an analytic model for blue
straggler formation in globular clusters.  Our model considers the
production of blue stragglers throughout the entire cluster, and
tracks their accumulation in the core due to dynamical friction (or,
equivalently, two-body relaxation).  Our results support the
conclusion that blue stragglers are descended from binary stars, as
first reported in \citet{knigge09} using the technique presented in
Chapter~\ref{chapter3} and Chapter~\ref{chapter4}.  Our model is applicable to
studying the radial distributions of other stellar and binary
populations in globular clusters.  This is a sorely needed theoretical
tool that can be used to address a number of recent observational 
studies that reported peculiarities among the radial distributions of
several different stellar and binary populations
\citep[e.g.][]{rood73, fusipecci93, ferraro04, lanzoni07}.

Finally, in Chapter~\ref{chapter6}, we extend the statistical
technique introduced in 
Chapter~\ref{chapter3} and Chapter~\ref{chapter4} to include the
entire main-sequence.  By applying our method to the ACS data, we have
obtained the first quantitative constraints for the degree of
universality of the stellar initial mass function for a large sample
of star clusters spanning a wider range of masses and chemical
compositions than ever before considered.  Given the very old ages of
the clusters in our sample, our results have important
implications for our understanding of star formation in the early
Universe.  Our results are 
consistent with a remarkably universal IMF in old
massive star clusters, and are well suited for
comparison to theoretical simulations of globular cluster evolution.
This would provide a simple means of directly quantifying
the extent to which the stellar mass function has been modified from
its primordial form by two-body relaxation.  In future studies, this
will allow us to obtain the very first constraints for the precise
functional form of the IMF and the degree of
primordial mass segregation in a large sample of old star clusters.

The results presented in this thesis have significantly advanced our
understanding of stellar dynamics, 
stellar evolution and, in particular, the interplay that occurs
between the two in dense star clusters.
%The results presented in this thesis represent a significant step
%forward in our understanding of stellar evolution, stellar dynamics,
%and the various interactions they undergo in star clusters.  
But we
are not yet done.  Our results have raised several important questions
related to these topics and, by extension, the history of our Galaxy.
Once again, we are left asking:  Where do we go from here?   For
example, how else might we use observations of blue straggler
populations to learn about the dynamical histories of their host
clusters?  How can we use this information to constrain the
interactions that occur in clusters between stellar dynamics and
binary star evolution?  How do these interactions affect the
observed properties of their binary populations and the various types
of stellar exotica they are home to?  There is also the issue of the
universality of the stellar 
initial mass function.  What combination of initial concentrations and
mass functions should have evolved dynamically over the lifetimes of
star clusters to reproduce the currently observed mass functions?
What can this tell us about primordial mass segregation in massive
star clusters, and the exact form of the stellar IMF?  All of these
issues relate to the over-arching questions:  How do stars
form?; and:  How did the history of our Galaxy unfold?  %These
%questions offer crucial insight into our understanding of
%star formation, and the evolution of our Galaxy.  
The next few years promise to hold exciting advances in the search for
answers to these questions.
% is currently underway, and will be the focus of future studies. 




%The application of our techniques to observations of star clusters
%have brought us to the following conclusions.  
%GO OVER KEY RESULTS HERE.  WHAT DID WE FIND?
%
%WHY IS WHAT WE FOUND IMPORTANT?  BOTH FOR THE SPECIFIC FIELDS OF
%STELLAR DYNAMICS AND STELLAR EVOLUTION (AND THEIR INTERACTION), BUT
%ALSO FOR THE BROADER IMPLICATIONS?  HOW CAN OUR METHODS BE EXTENDED
%AND APPLIED TO OTHER ASTROPHYSICAL/PHYSICAL PROBLEMS AND SCENARIOS?
%WHAT SIMILAR POPULATION STATISTICS AND THEORETICAL
%PROCESSES/TECHNIQUES (E.G. MEAN-FREE PATH APPROXIMATION) DO THEY ALSO
%APPLY TO?
%
%THEN, DIRECTIONS FOR FUTURE WORK.  WHAT HAS OUR WORK TOLD US ABOUT
%WHERE TO GO FROM HERE?
%
%
%
%%%Let us return to the analogy between stars and people with which
%%%we began this thesis.  
%Within stellar communities, the life of a star can be dramatically
%impacted by its peers.  The interactions are typically peaceful and
%develop gradually over time.  However, violent encounters also occur
%and what can be thought of as a shoving match ensues.  The skirmish ends
%abruptly when one or more of the stars are forcefully propelled from the
%encounter.  Either that, or stars are devoured by their opponents.
%The effects of all of these interactions reverberate throughout
%clusters, stimulating their 
%maturation.  Society evolves and adapts to accomodate the
%ever-changing demands its citizens impose upon it.  
%
%Theoretical work has painted a comprehensive picture for the various
%forms of stellar psychology operating within clusters, and their
%consequences.  The results of 
%this thesis have connected several of these processes to real
%observations of star clusters, in many cases for the first time.
%Given the importance of star clusters for stellar birthing, this 
%is a key step toward re-constructing the history of our Galaxy.
%What's more, it has allowed us to directly link the observed
%properties of several peculiar stellar populations to the physical
%processes responsible for their origins.
%
% COME BACK TO ANALOGY BETWEEN STARS AND PEOPLE HERE.

%\chapterbib

\begin{thebibliography}{99}



\bibitem[\protect\citeauthoryear{Abell}{1958}]{abell58} Abell
  G. O. 1958, ApJS, 3, 211
\bibitem[\protect\citeauthoryear{Abell, Corwin \&
    Olowin}{1989}]{abell89} Abell G. O., Corwin Jr. H. G., Olowin
  R. P. 1989, ApJS, 70, 1
\bibitem[\protect\citeauthoryear{Absil \& Mawet}{2010}]{absil10} Absil
  O., Mawet D. 2010, A\&ARv, 18, 317
\bibitem[\protect\citeauthoryear{del Burgo et al.}{2003}]{delburgo03}
  del Burgo C., Laureijs R. J., Abraham P., Kiss Cs., 2003, MNRAS,
  346, 403
\bibitem[\protect\citeauthoryear{Ferraro et al.}{2004}]{ferraro04}
  Ferraro F. R., Beccari G., Rood, R. T., Bellazzini M., Sills A.,
  Sabbi E. 2004, ApJ, 603, 127
\bibitem[\protect\citeauthoryear{Fusi Pecci et al.}{1993}]{fusipecci93}
  Fusi Pecci F., Ferraro F. R., Bellazzini M., et al. 1993, AJ, 105,
  1145
\bibitem[\protect\citeauthoryear{Glassgold}{1996}]{glassgold96}
  Glassgold A. E. 1996, ARA\&A, 34, 241
\bibitem[\protect\citeauthoryear{Kiss et al.}{2008}]{kiss06} Kiss Cs.,
  Abraham P., Laureijs R. J., Moor A., Birkmann S. M. 2006, MNRAS,
  373, 1213
\bibitem[\protect\citeauthoryear{Knigge, Leigh \&
    Sills}{2009}]{knigge09} Knigge C., Leigh N., Sills A. 2009,
  Nature, 457, 288
\bibitem[\protect\citeauthoryear{Lanzoni et al.}{2007}]{lanzoni07}
  Lanzoni B., Dalessandro E., Perina S., Ferraro F. R., Rood R. T.,
  Sollima A. 2007, ApJ, 670, 1065
\bibitem[\protect\citeauthoryear{McKee \& Ostriker}{2007}]{mckee07}
  McKee C. F., Ostriker E. C. 2007, ARA\&A, 45, 565
\bibitem[\protect\citeauthoryear{Rood}{1973}]{rood73} Rood
  R. T. 1973, ApJ, 184, 815
\bibitem[\protect\citeauthoryear{Spitzer}{1941a}]{spitzer41a}
  Spitzer L. Jr. 1941, ApJ, 93, 369
\bibitem[\protect\citeauthoryear{Spitzer}{1941b}]{spitzer41b}
  Spitzer L. Jr. 1941, ApJ, 94, 232
\bibitem[\protect\citeauthoryear{Spitzer}{1942}]{spitzer42}
  Spitzer L. Jr. 1942, ApJ, 95, 329

\end{thebibliography}

%\bsp

\label{lastpage}
