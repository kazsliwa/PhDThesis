\pagestyle{fancy}
\headheight 20pt
\lhead{PhD Thesis - N. Leigh }
\rhead{McMaster Physics and Astronomy}
\chead{}
\lfoot{}
\cfoot{\thepage}
\rfoot{}
\renewcommand{\headrulewidth}{0.1pt}
\renewcommand{\footrulewidth}{0.1pt}

\chapter{Appendix B} \label{Appendix-B}
\thispagestyle{fancy}

In this appendix, we present our selection criteria for BS, RGB, HB
and MSTO stars used in Chapter~\ref{chapter4}.  Our method is similar
to that described in \citet{leigh07},
and we have used this as a basis for the selection criteria presented
in this chapter.  First, we define a location for the MSTO in the
(F606W-F814W)-F814W plane using our isochrone fits.  The MSTO is
chosen to be the bluest point along the MS of each isochrone, which we
denote by ((V-I)$_{MSTO}$,I$_{MSTO}$).  In order to distinguish BSs
from MSTO stars, we impose the conditions:

\begin{equation}
\label{eqn:bs_msto1}
F814W \le m_1(F606W-F814W) + b_{11},
\end{equation}
where the slope of this line is $m_1 = -9$ and its y-intercept is
given by:
\begin{equation}
\label{eqn:b_11}
b_{11} = (I_{MSTO}-0.10)-m_1((V-I)_{MSTO}-0.10)
\end{equation}

Similarly, we distinguish BSs from HB stars by defining the following
additional boundaries:
\begin{eqnarray}
\label{bs_hb1}
F814W &\ge& m_1(F606W-F814W) + b_{12} \\
F814W &\ge& m_2(F606W-F814W) + b_{21} \\
F814W &\le& m_2(F606W-F814W) + b_{22} \\
F814W &\ge& m_{HB}(F606W-F814W) + b_{HB} \\
(F606W-F814W) &\ge& (V-I)_{HB} \\
F814W &\le& I_{MSTO},
\end{eqnarray}
where $m_2 = 6$, $m_{HB} = -1.5$ and $(V-I)_{HB} = (V-I)_{MSTO} -
0.4$.  We also define:
\begin{eqnarray}
\label{b_12}
b_{12} &=& (I_{MSTO}-0.55)-m_1((V-I)_{MSTO}-0.55) \\
b_{21} &=& (I_{MSTO}-0.80)-m_2((V-I)_{MSTO}+0.10) \\
b_{22} &=& (I_{MSTO}+0.30)-m_2((V-I)_{MSTO}-0.20),
\end{eqnarray}
and $b_{HB} = I_{HB} + 1.2$, where $I_{HB}$ roughly corresponds to the
mid-point of points that populate the HB and is chosen
by eye for each cluster so that our selection criteria best fits the
HB in all of the CMDs in our sample.

We apply a similar set of conditions to the RGB in order to select
stars belonging to this stellar population.  These boundary conditions
are:
\begin{eqnarray}
\label{rgb_1}
F814W &\ge& m_{HB}(F606W-F814W) + b_{HB} \\
F814W &\ge& m_{RGB}(F606W-F814W) + b_{31} \\
F814W &\le& m_{RGB}(F606W-F814W) + b_{32} \\
F814W &\le& I_{RGB},
\end{eqnarray}
where $m_{RGB} = -23$, $I_{RGB}$ is defined as the F814W magnitude
corresponding to a core helium mass of 0.08 M$_{\odot}$ and:
\begin{eqnarray}
\label{b_31_and_b_32}
b_{31} &=& (I_{MSTO}-0.60)-m_{RGB}((V-I)_{MSTO}+0.05) \\
b_{32} &=& (I_{MSTO}-0.60)-m_{RGB}((V-I)_{MSTO}+0.25)
\end{eqnarray}

Core helium-burning stars, which we refer to as HB stars, are selected
if they satisfy one of the following sets of criteria:
\begin{eqnarray}
\label{hb1}
F814W &\ge& m_{HB}(F606W-F814W) + (b_{HB}-1.0) \\
F814W &\le& m_{HB}(F606W-F814W) + b_{HB} \\
(F606W-F814W) &\le& (V-I)_{MSTO} + (V-I)_{HB},
\end{eqnarray}
\begin{eqnarray}
\label{hb2}
F814W &>& m_{HB}(F606W-F814W) + b_{HB} \\
F814W &\le& I_{MSTO} + 2.5 \\
(F606W-F814W) &<& (V-I)_{MSTO} - 0.4,
\end{eqnarray}
or
\begin{eqnarray}
\label{hb3}
F814W &<& m_1(F606W-F814W) + b_{12} \\
F814W &>& m_{HB}(F606W-F814W) + b_{HB} \\
(F606W-F814W) &\ge& (V-I)_{MSTO} - 0.4
\end{eqnarray}
We define $(V-I)_{HB}$ on a cluster-by-cluster basis in order to
ensure that we do not over- or under-count the number of HB stars.
This is because the precise value of (F606W-F814W) at which the HB
becomes the RGB varies from cluster-to-cluster.  In addition, the
precise location of the transition in the cluster CMD between HB
and EHB stars remains poorly understood.  To avoid this ambiguity, we
consider HB and EHB stars
together throughout our analysis, and collectively refer to all core
helium-burning stars as HB stars throughout this chapter.

Finally, MSTO stars are selected according to the following criteria:
\begin{eqnarray}
\label{msto_1}
F814W &>& I_{RGB} \\
F814W &>& m_1(F606W-F814W) + b_{11} \\
F814W &\le& (V-I)_{MSTO}
\end{eqnarray}
