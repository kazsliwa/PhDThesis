\pagestyle{fancy}
\headheight 20pt
\lhead{PhD Thesis - N. Leigh }
\rhead{McMaster Physics and Astronomy}
\chead{}
\lfoot{}
\cfoot{\thepage}
\rfoot{}
\renewcommand{\headrulewidth}{0.1pt}
\renewcommand{\footrulewidth}{0.1pt}

\chapter{Appendix A} \label{Appendix-A} 
\thispagestyle{fancy} 

In this appendix, we present the collisional cross sections and
time-scales used in Chapter~\ref{chapter2}.  The
gravitationally-focused cross sections for 1+1,
1+2, 2+2, 1+3, 2+3 and 3+3 collisions can be found using
Equation 6 from \citet{leonard89}.  Neglecting the first term and
assuming that binary and triple stars are on average twice and three
times as massive as single stars, respectively, this gives for the
various collisional cross sections:
\begin{equation}
\label{eqn:cs-1+1}
\sigma_{1+1} \sim \frac{8{\pi}GmR}{v_{rel}^2},
\end{equation}
\begin{equation}
\label{eqn:cs-1+2}
\sigma_{1+2} \sim \frac{3{\pi}Gma_b}{v_{rel}^2},
\end{equation}
\begin{equation}
\label{eqn:cs-2+2}
\sigma_{2+2} \sim \frac{8{\pi}Gma_b}{v_{rel}^2},
\end{equation}
\begin{equation}
\label{eqn:cs-1+3}
\sigma_{1+3} \sim \frac{4{\pi}Gma_t}{v_{rel}^2},
\end{equation}
\begin{equation}
\label{eqn:cs-2+3}
\sigma_{2+3} \sim \frac{5{\pi}Gm(a_b + a_t)}{v_{rel}^2},
\end{equation}
\begin{equation}
\label{eqn:cs-3+3}
\sigma_{3+3} \sim \frac{12{\pi}Gma_t}{v_{rel}^2}.
\end{equation}
Values for the
pericenters assumed for the various types of encounters are shown in
Table~\ref{table:peri}, where $R$ is the average stellar
radius, $a_b$ is the average binary semi-major axis and $a_t$ is the
average semi-major axis of the outer orbits of triples.

\begin{table}
\centering
\caption{Pericenters Assumed for Each Encounter Type
  \label{table:peri}}
\begin{tabular}{lc}
\hline
Encounter Type & Pericenter \\
\hline
1+1 & $2R$ \\
1+2 & $a_b/2$ \\
1+3 & $a_t/2$ \\
2+2 & $a_b$ \\
2+3 & $(a_b + a_t)/2$ \\
3+3 & $a_t$ \\
\hline
\end{tabular}
\end{table}

In general, the time between each of the different encounter
types can be found using Equation~\ref{eqn:coll-rate} and
the gravitationally-focused cross sections for collision given
above.  Following the derivation of \citet{leonard89}, we can write
the encounter rate in the general form:
\begin{equation}
\label{eqn:coll-rate-gen}
\Gamma_{x+y} = N_xn_y{\sigma}_{x+y}v_{x+y},
\end{equation}
where $N_x$ and $n_y$ are the number and number density, respectively,
of single, binary or triple stars and $v_{x+y}$ is the relative
velocity at infinity between objects $x$ and $y$.  For instance, the
time between binary-binary encounters in the core of a cluster is
given by:
\begin{equation}
\begin{gathered}
\label{eqn:coll2+2}
\tau_{2+2} = 1.3 \times 10^7f_b^{-2} \Big(\frac{1
  pc}{r_c}\Big)^3\Big(\frac{10^3
  pc^{-3}}{n_0}\Big)^2 \\
\Big(\frac{v_{rms}}{5 km/s}\Big)\Big(\frac{0.5
  M_{\odot}}{<m>}\Big)\Big(\frac{1
  AU}{a_{b}} \Big) \mbox{ years}.
\end{gathered}
\end{equation}
Similarly, the times between 1+1, 1+2, 1+3, 2+3 and 3+3 encounters are
given by:
\begin{equation}
\begin{gathered}
\label{eqn:coll1+1}
\tau_{1+1} = 1.1 \times 10^{10}(1-f_b-f_t)^{-2}\Big(\frac{1 pc}{r_c}
\Big)^3 \Big(\frac{10^3 pc^{-3}}{n_0} \Big)^2 \\
\Big(\frac{v_{rms}}{5 km/s} \Big) \Big(\frac{0.5 M_{\odot}}{<m>} \Big)
\Big(\frac{0.5 R_{\odot}}{<R>} \Big)\mbox{ years},
\end{gathered}
\end{equation}
\begin{equation}
\begin{gathered}
\label{eqn:coll1+2}
\tau_{1+2} = 3.4 \times 10^7(1-f_b-f_t)^{-1}f_b^{-1} \Big(\frac{1
  pc}{r_c}
\Big)^3 \Big(\frac{10^3 pc^{-3}}{n_0} \Big)^2 \\
\Big(\frac{v_{rms}}{5
  km/s} \Big) \Big(\frac{0.5 M_{\odot}}{<m>} \Big) \Big(\frac{1
  AU}{a_{b}} \Big)\mbox{ years},
\end{gathered}
\end{equation}
\begin{equation}
\begin{gathered}
\label{eqn:coll1+3}
\tau_{1+3} = 2.6 \times 10^7(1-f_b-f_t)^{-1}f_t^{-1} \Big(\frac{1
  pc}{r_c}
\Big)^3 \Big(\frac{10^3 pc^{-3}}{n_0} \Big)^2 \\
\Big(\frac{v_{rms}}{5
  km/s} \Big) \Big(\frac{0.5 M_{\odot}}{<m>} \Big) \Big(\frac{1
  AU}{a_{t}} \Big)\mbox{ years},
\end{gathered}
\end{equation}
\begin{equation}
\begin{gathered}
\label{eqn:coll2+3}
\tau_{2+3} = 2.0 \times 10^7f_b^{-1}f_t^{-1} \Big(\frac{1 pc}{r_c}
\Big)^3 \Big(\frac{10^3 pc^{-3}}{n_0} \Big)^2 \\
\Big(\frac{v_{rms}}{5
  km/s} \Big) \Big(\frac{0.5 M_{\odot}}{<m>} \Big) \Big(\frac{1
  AU}{a_{b}+a_{t}} \Big)\mbox{ years},
\end{gathered}
\end{equation}
and
\begin{equation}
\begin{gathered}
\label{eqn:coll3+3}
\tau_{3+3} = 8.3 \times 10^6f_t^{-2} \Big(\frac{1 pc}{r_c}
\Big)^3 \Big(\frac{10^3 pc^{-3}}{n_0} \Big)^2 \\
\Big(\frac{v_{rms}}{5
  km/s} \Big) \Big(\frac{0.5 M_{\odot}}{<m>} \Big) \Big(\frac{1
  AU}{a_{t}} \Big)\mbox{ years}.
\end{gathered}
\end{equation}


